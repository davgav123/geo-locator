\documentclass[a4paper]{article}
\usepackage[OT1, OT2]{fontenc}
\setlength{\textheight}{25cm}
\setlength{\textwidth}{18cm}
\setlength{\topmargin}{-25mm}
\setlength{\hoffset}{-25mm}
\def\zn{,\kern-0.09em,}

\newcommand{\Lat}{\fontencoding{OT1}\fontfamily{cmr}\selectfont}

\begin{document}
\thispagestyle{empty}

\fontfamily{wncyr}
\fontencoding{OT2}\selectfont

\begin{flushleft}
Matematichki fakultet\\
Univerziteta u Beogradu
\end{flushleft}

\bigskip

\begin{center}
\textbf{MOLBA\\
ZA ODOBRAVANJE TEME MASTER RADA
}\end{center}

\bigskip

\begin{flushleft}
Molim da se odobri izrada master rada pod naslovom:
\end{flushleft}

\begin{minipage}{16.5cm}
%%%%%%%%%%%%%%%%%%%%%%%%%%%%%%%%%%%%%%%%%%%%%%%%%%%%%%%%%%%%%%%%%%%%%%%%%%%%%%%
% U donji red upisati naziv master rada umesto teksta: "Naziv master rada"    %
%%%%%%%%%%%%%%%%%%%%%%%%%%%%%%%%%%%%%%%%%%%%%%%%%%%%%%%%%%%%%%%%%%%%%%%%%%%%%%%
\textbf{\textit{\zn Distribuirana obrada geoprostornih podataka''}}
\end{minipage}\\
\rule[4mm]{17.5cm}{.05mm}
\begin{flushleft}
\framebox{
\begin{minipage}[t][10.8cm]{17cm}
%%%%%%%%%%%%%%%%%%%%%%%%%%%%%%%%%%%%%%%%%%%%%%%%%%%%%%%%%%%%%%%%%%%%%%%%%%%%%%%
%  -- unutrasnjost pravougaonika --                                           %
%  Umesto donjeg teksta upisati znacaj i specificni cilj master rada          %
%%%%%%%%%%%%%%%%%%%%%%%%%%%%%%%%%%%%%%%%%%%%%%%%%%%%%%%%%%%%%%%%%%%%%%%%%%%%%%%
\textbf{Znachaj teme i oblasti:}

Svakodnevno se generishe velika kolichina podataka. Postoji veliki broj primena ovih podataka, na primer, razne vrste analitika i pravljenje modela mashinskog uchenja. Obrada podataka se најчешће izvrshava коришћењем distribuiranih sistema. Distribuirani sistemi su sachinjeni od umrezhenih mashina koje medjusobno saradjuju i paralelno izvrshavaju posao. Primer takvog sistema je fajl sistem HDFS, centralni deo Hadup ekosistema [1]. Obrade velikih kolichina podataka se ostvaruju podelom podataka na delove koji se zatim paralelno obradjuju na odvojenim mashinama. U distribuiranoj obradi podataka je zastupljen programski jezik Skala [2] u kome su napisani alati poput Sparka [3], Kafke i Flinka.

Geoprostorni podaci su podaci koji predstavljaju lokacije na geografskoj mapi, kao i opis tih lokacija. Pored toga shto se koriste u aplikacijama kojima je primarna svrha rad sa mapama, koriste se i za optimalan raspored raznih vrsta infrastrukura, zdravstvenih centara i slichno. Takodje, kombinovanjem sa drugim podacima se mogu koristiti za predvidjanje vremenskih prilika kao i za razne geografske vizualizacije. Jedan primer geoprostornih podataka otvorenog koda je OSM [4].

\textbf{Specifichni cilj rada:}

U okviru rada bic1e prikazan nachin rada distribuiranih sistema i tehnologija koje se koriste za obradu velike kolichine podataka. Takodje, bic1e implementirana aplikacija za rad sa geoprostornim podacima koja se mozhe koristiti radi boljeg iskustva korisnika koji koriste geografske mape. Aplikacija c1e koristiti programski jezik Skala i Spark biblioteku. Za prikaz rezultata obrade podataka c1e se koristiti programski jezik Javaskript.

\textbf{Literatura:}

\Lat{[1] \textit{Garry Turkington, Gabriele Modena}. \textit{Learning Hadoop 2}. 2015.

[2] \textit{Lex Spoon, Martin Odersky, Bill Venners}. \textit{Programming in Scala, First edition.} 2008.

[3] \textit{Bill Chambers, Matei Zaharia}. \textit{Spark: The definitive guide.} 2018.

[4] \textit{OpenStreetMap Foundation}. \textit{Openstreetmap wiki}. online at: https://wiki.openstreetmap.org/.
}

%\begin{tabular}{|c|c|}
%    \hline
%    \multicolumn{2}{|c|}{\textbf{Uput{}stvo za pisanje nashih slova}} \\
%    \hline\hline
%	ligatura & rezultujuc1i simbol  \\
%    \hline
%    \texttt{{\Lat dj}} & dj \\
%    \hline
%    \texttt{{\Lat Dj}} & Dj \\
%    \hline
%    \texttt{{\Lat zh}} & zh \\
%    \hline
%    \texttt{{\Lat Zh}} & Zh \\
%    \hline
%    \texttt{{\Lat lj}} & lj \\
%    \hline
%    \texttt{{\Lat Lj}} & Lj \\
%    \hline
%    \texttt{{\Lat nj}} & nj \\
%    \hline
%    \texttt{{\Lat Nj}} & Nj \\
%    \hline
%    \texttt{{\Lat c1}} & c1 \\
%    \hline
%    \texttt{{\Lat C1}} & C1 \\
%    \hline
%    \texttt{{\Lat ch}} & ch \\
%    \hline
%    \texttt{{\Lat Ch}} & Ch \\
%    \hline
%    \texttt{{\Lat d2}} & d2 \\
%    \hline
%    \texttt{{\Lat D2}} & D2 \\
%    \hline
%    \texttt{{\Lat sh}} & sh \\
%    \hline
%    \texttt{{\Lat Sh}} & Sh \\
%    \hline
%    \texttt{{\Lat ts}} & ts \\
%    \hline
%    \texttt{{\Lat t\{\}s}} & t{}s \\
%    \hline
%\end{tabular}


\end{minipage}
}
\end{flushleft}
\vspace{1cm}
%%%%%%%%%%%%%%%%%%%%%%%%%%%%%%%%%%%%%%%%%%%%%%%%%%%%%%%%%%%%%%%%%%%%%%%%%%%%%%%
% u donji red uneti:         ime i prezime, broj indeksa i modul studenta     %
%%%%%%%%%%%%%%%%%%%%%%%%%%%%%%%%%%%%%%%%%%%%%%%%%%%%%%%%%%%%%%%%%%%%%%%%%%%%%%%
\makebox[9cm][c]{\textbf{David Gavrilovic1, 1100/2019, Информатика}}
%%%%%%%%%%%%%%%%%%%%%%%%%%%%%%%%%%%%%%%%%%%%%%%%%%%%%%%%%%%%%%%%%%%%%%%%%%%%%%%
% u donji red uneti:               ime i prezime mentora                      %
%%%%%%%%%%%%%%%%%%%%%%%%%%%%%%%%%%%%%%%%%%%%%%%%%%%%%%%%%%%%%%%%%%%%%%%%%%%%%%%
Saglasan mentor \makebox[6cm][c]{\textbf{dr Milena Vujoshevic1 Janichic1}} \\
\rule[4mm]{9cm}{.05mm} \hfill \raisebox{4mm}{\makebox[6cm][l]{.\dotfill.}} \\
\raisebox{1cm}%
[9mm][0mm]{\makebox[10cm][c]{\textit{(ime i prezime studenta, br. indeksa, smer i modul)}}} \\
\makebox[10cm]{ }\\
\vspace{-1cm}\\
\rule[2cm]{6.5cm}{.05mm} \hfill \rule[2cm]{6.5cm}{.05mm}\\
\vspace{-2.4cm}\\
\raisebox{2cm}{\makebox[6.5cm][c]{\textit{(svojeruchni potpis studenta)}}}
\hfill \raisebox{2cm}{\makebox[6.5cm][c]{\textit{(svojeruchni potpis mentora)}}}\\
\vspace{-2cm}\\
%%%%%%%%%%%%%%%%%%%%%%%%%%%%%%%%%%%%%%%%%%%%%%%%%%%%%%%%%%%%%%%%%%%%%%%%%%%%%%%
% u donji red uneti datum podnosenja molbe                                    %
%%%%%%%%%%%%%%%%%%%%%%%%%%%%%%%%%%%%%%%%%%%%%%%%%%%%%%%%%%%%%%%%%%%%%%%%%%%%%%%
\makebox[5.5cm][c]{10.5.2022.}\makebox[5.5cm]{} Chlanovi komisije\\
%%%%%%%%%%%%%%%%%%%%%%%%%%%%%%%%%%%%%%%%%%%%%%%%%%%%%%%%%%%%%%%%%%%%%%%%%%%%%%%
% POPUNJAVA MENTOR (rucno ili na sledeci nacin):                              %
% u donji red umesto .dotfill. upisati podatke o 1. clanu komisije            %
%%%%%%%%%%%%%%%%%%%%%%%%%%%%%%%%%%%%%%%%%%%%%%%%%%%%%%%%%%%%%%%%%%%%%%%%%%%%%%%
\rule[4mm]{5.5cm}{.05mm}\makebox[5.5cm]{ } 1. \makebox[6cm][l]{dr Mirko Spasic1}\\
\vspace{-8mm}\\
\raisebox{4mm}%
[7mm][0mm]{\makebox[5.5cm][c]{\textit{(datum podnoshenja molbe)}}}\makebox[5.5cm]{ }
%%%%%%%%%%%%%%%%%%%%%%%%%%%%%%%%%%%%%%%%%%%%%%%%%%%%%%%%%%%%%%%%%%%%%%%%%%%%%%%
% POPUNJAVA MENTOR (rucno ili na sledeci nacin):                              %
% u donji red umesto .\dotfill. upisati podatke o 2. clanu komisije           %
%%%%%%%%%%%%%%%%%%%%%%%%%%%%%%%%%%%%%%%%%%%%%%%%%%%%%%%%%%%%%%%%%%%%%%%%%%%%%%%
2. \makebox[6cm][l]{dr Sasha Malkov}\\

\vspace{1cm}


\begin{flushleft}
%%%%%%%%%%%%%%%%%%%%%%%%%%%%%%%%%%%%%%%%%%%%%%%%%%%%%%%%%%%%%%%%%%%%%%%%%%%%%%%
% u donji red upisati                 katedru                                 %
%%%%%%%%%%%%%%%%%%%%%%%%%%%%%%%%%%%%%%%%%%%%%%%%%%%%%%%%%%%%%%%%%%%%%%%%%%%%%%%
Katedra \makebox[9.5cm][l]{za rachunarstvo i informatiku} je saglasna sa predlozhenom temom.
\vspace{-3mm}
\hspace*{13mm} \rule[2.3cm]{9.5cm}{.05mm}\\
\vspace{-1cm}
%%%%%%%%%%%%%%%%%%%%%%%%%%%%%%%%%%%%%%%%%%%%%%%%%%%%%%%%%%%%%%%%%%%%%%%%%%%%%%%
% POPUNJAVA SEF KATEDRE                                                       %
%%%%%%%%%%%%%%%%%%%%%%%%%%%%%%%%%%%%%%%%%%%%%%%%%%%%%%%%%%%%%%%%%%%%%%%%%%%%%%%
\makebox[6.5cm][c]{} \hfill \makebox[6.5cm][c]{}\\
\rule[4mm]{6.5cm}{.05mm} \hfill \rule[4mm]{6.5cm}{.05mm}\\
\vspace{-5mm}
\makebox[6.5cm][c]{\textit{(shef katedre)}} \hfill \makebox[6.5cm][c]{\textit{(datum odobravanja molbe)}}
\end{flushleft}
\end{document} 
